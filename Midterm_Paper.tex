\documentclass[man]{apa6}
\usepackage{lmodern}
\usepackage{amssymb,amsmath}
\usepackage{ifxetex,ifluatex}
\usepackage{fixltx2e} % provides \textsubscript
\ifnum 0\ifxetex 1\fi\ifluatex 1\fi=0 % if pdftex
  \usepackage[T1]{fontenc}
  \usepackage[utf8]{inputenc}
\else % if luatex or xelatex
  \ifxetex
    \usepackage{mathspec}
  \else
    \usepackage{fontspec}
  \fi
  \defaultfontfeatures{Ligatures=TeX,Scale=MatchLowercase}
\fi
% use upquote if available, for straight quotes in verbatim environments
\IfFileExists{upquote.sty}{\usepackage{upquote}}{}
% use microtype if available
\IfFileExists{microtype.sty}{%
\usepackage{microtype}
\UseMicrotypeSet[protrusion]{basicmath} % disable protrusion for tt fonts
}{}
\usepackage{hyperref}
\hypersetup{unicode=true,
            pdftitle={Reanalysis of Psychological Paper: Computer Game Play Reduces Intrusive Memories of Experimental Trauma via Reconsolidation-Update Mechanisms},
            pdfauthor={Ana-Louise Franz},
            pdfkeywords={reconsolidation, cognitive task},
            pdfborder={0 0 0},
            breaklinks=true}
\urlstyle{same}  % don't use monospace font for urls
\usepackage{graphicx,grffile}
\makeatletter
\def\maxwidth{\ifdim\Gin@nat@width>\linewidth\linewidth\else\Gin@nat@width\fi}
\def\maxheight{\ifdim\Gin@nat@height>\textheight\textheight\else\Gin@nat@height\fi}
\makeatother
% Scale images if necessary, so that they will not overflow the page
% margins by default, and it is still possible to overwrite the defaults
% using explicit options in \includegraphics[width, height, ...]{}
\setkeys{Gin}{width=\maxwidth,height=\maxheight,keepaspectratio}
\IfFileExists{parskip.sty}{%
\usepackage{parskip}
}{% else
\setlength{\parindent}{0pt}
\setlength{\parskip}{6pt plus 2pt minus 1pt}
}
\setlength{\emergencystretch}{3em}  % prevent overfull lines
\providecommand{\tightlist}{%
  \setlength{\itemsep}{0pt}\setlength{\parskip}{0pt}}
\setcounter{secnumdepth}{0}
% Redefines (sub)paragraphs to behave more like sections
\ifx\paragraph\undefined\else
\let\oldparagraph\paragraph
\renewcommand{\paragraph}[1]{\oldparagraph{#1}\mbox{}}
\fi
\ifx\subparagraph\undefined\else
\let\oldsubparagraph\subparagraph
\renewcommand{\subparagraph}[1]{\oldsubparagraph{#1}\mbox{}}
\fi

%%% Use protect on footnotes to avoid problems with footnotes in titles
\let\rmarkdownfootnote\footnote%
\def\footnote{\protect\rmarkdownfootnote}


  \title{Reanalysis of Psychological Paper: Computer Game Play Reduces Intrusive
Memories of Experimental Trauma via Reconsolidation-Update Mechanisms}
    \author{Ana-Louise Franz\textsuperscript{}}
    \date{}
  
\shorttitle{Reanalysis}
\affiliation{
\vspace{0.5cm}
\textsuperscript{1} Brooklyn College}
\keywords{reconsolidation, cognitive task\newline\indent Word count: X}
\usepackage{csquotes}
\usepackage{upgreek}
\captionsetup{font=singlespacing,justification=justified}

\usepackage{longtable}
\usepackage{lscape}
\usepackage{multirow}
\usepackage{tabularx}
\usepackage[flushleft]{threeparttable}
\usepackage{threeparttablex}

\newenvironment{lltable}{\begin{landscape}\begin{center}\begin{ThreePartTable}}{\end{ThreePartTable}\end{center}\end{landscape}}

\makeatletter
\newcommand\LastLTentrywidth{1em}
\newlength\longtablewidth
\setlength{\longtablewidth}{1in}
\newcommand{\getlongtablewidth}{\begingroup \ifcsname LT@\roman{LT@tables}\endcsname \global\longtablewidth=0pt \renewcommand{\LT@entry}[2]{\global\advance\longtablewidth by ##2\relax\gdef\LastLTentrywidth{##2}}\@nameuse{LT@\roman{LT@tables}} \fi \endgroup}


\DeclareDelayedFloatFlavor{ThreePartTable}{table}
\DeclareDelayedFloatFlavor{lltable}{table}
\DeclareDelayedFloatFlavor*{longtable}{table}
\makeatletter
\renewcommand{\efloat@iwrite}[1]{\immediate\expandafter\protected@write\csname efloat@post#1\endcsname{}}
\makeatother
\usepackage{lineno}

\linenumbers

\authornote{

Correspondence concerning this article should be addressed to Ana-Louise
Franz, Postal address. E-mail:
\href{mailto:afranz100@gmail.com}{\nolinkurl{afranz100@gmail.com}}}

\abstract{
There are a few moments in the creation and recollection of memory where
this process can be interrupted. This can be used to help people who are
suffering from the results of tramatic memories. This study examined the
process of reconsolidation, the recollection of a memory, to determine
if there is a way to inturrupt this process using a cognitive task. The
cognitive task used in this experiment was a simple game of Tetris.


}

\begin{document}
\maketitle

\section{Methods}\label{methods}

\subsection{Participants}\label{participants}

52 participants (31 female, 21 males) which consisted of university
students and the general public. 65\% of the participants were students.

\subsection{Material}\label{material}

The details of the trauma exposure and the reconsolidation task are
detailed in James et al. (2015)

\subsection{Procedure}\label{procedure}

The experiment was performed both in the lab and at home in the form of
a diary. They watched a traumatic film and were then assinged to either
the cognitive task group or the no task (control) group.

\subsection{Data analysis}\label{data-analysis}

We used R (Version 3.5.2; R Core Team, 2018) and the R-packages
\emph{data.table} (Version 1.12.2; Dowle \& Srinivasan, 2019),
\emph{devtools} (Version 2.0.1; Wickham, Hester, \& Chang, 2018),
\emph{dplyr} (Version 0.8.0.1; Wickham, François, Henry, \& Müller,
2019), \emph{ggplot2} (Version 3.1.0; Wickham, 2016), \emph{papaja}
(Version 0.1.0.9842; Aust \& Barth, 2018), \emph{summarytools} (Version
0.9.2; Comtois, 2019), \emph{usethis} (Version 1.4.0; Wickham \& Bryan,
2018), and \emph{xtable} (Version 1.8.3; Dahl, Scott, Roosen, Magnusson,
\& Swinton, 2018) for all our analyses.

\section{Results}\label{results}

\begin{figure}
\centering
\includegraphics{Midterm_Paper_files/figure-latex/unnamed-chunk-1-1.pdf}
\caption{}
\end{figure}

Using a between subjects one-factor ANOVA, with intervention type as the
independent variable, I did not find that there was a significant
difference between the four intervention groups (No-task control,
Reactivation Plus tetris, Tetris only, Reactivation only). There was no
main effect of intervention type \(F(1, 70) = 0.11\),
\(\mathit{MSE} = 11.42\), \(p = .744\), \(\hat{\eta}^2_G = .002\).

\section{Discussion}\label{discussion}

\newpage

\section{References}\label{references}

\begingroup
\setlength{\parindent}{-0.5in} \setlength{\leftskip}{0.5in}

\hypertarget{refs}{}
\hypertarget{ref-R-papaja}{}
Aust, F., \& Barth, M. (2018). \emph{papaja: Create APA manuscripts with
R Markdown}. Retrieved from \url{https://github.com/crsh/papaja}

\hypertarget{ref-R-summarytools}{}
Comtois, D. (2019). \emph{Summarytools: Tools to quickly and neatly
summarize data}. Retrieved from
\url{https://CRAN.R-project.org/package=summarytools}

\hypertarget{ref-R-xtable}{}
Dahl, D. B., Scott, D., Roosen, C., Magnusson, A., \& Swinton, J.
(2018). \emph{Xtable: Export tables to latex or html}. Retrieved from
\url{https://CRAN.R-project.org/package=xtable}

\hypertarget{ref-R-data.table}{}
Dowle, M., \& Srinivasan, A. (2019). \emph{Data.table: Extension of
`data.frame`}. Retrieved from
\url{https://CRAN.R-project.org/package=data.table}

\hypertarget{ref-R-base}{}
R Core Team. (2018). \emph{R: A language and environment for statistical
computing}. Vienna, Austria: R Foundation for Statistical Computing.
Retrieved from \url{https://www.R-project.org/}

\hypertarget{ref-R-ggplot2}{}
Wickham, H. (2016). \emph{Ggplot2: Elegant graphics for data analysis}.
Springer-Verlag New York. Retrieved from \url{http://ggplot2.org}

\hypertarget{ref-R-usethis}{}
Wickham, H., \& Bryan, J. (2018). \emph{Usethis: Automate package and
project setup}. Retrieved from
\url{https://CRAN.R-project.org/package=usethis}

\hypertarget{ref-R-dplyr}{}
Wickham, H., François, R., Henry, L., \& Müller, K. (2019). \emph{Dplyr:
A grammar of data manipulation}. Retrieved from
\url{https://CRAN.R-project.org/package=dplyr}

\hypertarget{ref-R-devtools}{}
Wickham, H., Hester, J., \& Chang, W. (2018). \emph{Devtools: Tools to
make developing r packages easier}. Retrieved from
\url{https://CRAN.R-project.org/package=devtools}

\endgroup


\end{document}
